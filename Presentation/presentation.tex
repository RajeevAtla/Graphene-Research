\documentclass[12pt]{beamer}
\usepackage[utf8]{inputenc}
\usepackage{graphicx}

\graphicspath{ {./images/} }



\usetheme{PaloAlto}
\usecolortheme{rose}


\title{Exploring Superconductors Using Generative Adversarial Networks}
\author[Rajeev Atla]
{Rajeev Atla}

\institute[JPS]
{
  John P. Stevens High School
}
\begin{document}
\frame{\titlepage}



\section{Introduction}
\subsection{Superconductivity}

\begin{frame}
\frametitle{Outline}
\tableofcontents[currentsection]
\end{frame}






\begin{frame}
\frametitle{Introduction}
\begin{itemize}
  \pause
  \item Superconductors have the potential to change the world
  \pause
  \item Power transmission
  \pause
  \item Temperature
  \pause
  \item Investigation
  \pause
  \begin{itemize}
      \item Trial-and-error process
      \pause
      \item Labor and resource-intensive
      \pause
      \item Every material's phase diagram must be completely mapped out
  \end{itemize}
\end{itemize}
\end{frame}





\section{Objective}
\begin{frame}
\frametitle{Outline}
\tableofcontents[currentsection]
\end{frame}

\begin{frame}
\frametitle{Objective}

\begin{itemize}
    \item Shorten the amount of time necessary to find a novel superconductor
    \pause
    \item Allow researchers to inform their selection of a superconductor for experimental verification
\end{itemize}

\end{frame}


\section{Data}
\begin{frame}
\frametitle{Outline}
\tableofcontents[currentsection]
\end{frame}

\begin{frame}
\frametitle{Data}

\begin{itemize}
  \item Taken from UCI (University of California, Irvine) Machine Learning Repository
  \pause
  \item 21,263 examples with 81 features
  \pause
\end{itemize}

\begin{figure}[h]
  \includegraphics[scale = 0.5]{UCIRepo.png}
  \caption{\textbf{UCI Machine Learning Repository}}
\end{figure}

\end{frame}



\section{Methods}
\begin{frame}
\frametitle{Outline}
\tableofcontents[currentsection]
\end{frame}



\begin{frame}
\frametitle{Methods}

\pause
\begin{itemize}
    \item Generative adversial netowrk
    \pause
    \item Generator uses the data given to it to create a model that can output data
    \pause
    \item The discriminator "checks" this data
    \pause
    \item The two models train each other
\end{itemize}

\end{frame}

\section{Results}
\begin{frame}
\frametitle{Outline}
\tableofcontents[currentsection]
\end{frame}

\begin{frame}

\frametitle{Results}
\begin{itemize}
    \item TODO
\end{itemize}

\end{frame}

\section{Discussion}
\begin{frame}
\frametitle{Outline}
\tableofcontents[currentsection]
\end{frame}

\begin{frame}
\frametitle{Discussion}
\pause
\begin{itemize}
  \item Data was generated by the GAN
  \pause
  \item This data can now be tested
\end{itemize}
\pause

\end{frame}

\section{Conclusion}
\begin{frame}
\frametitle{Outline}
\tableofcontents[currentsection]
\end{frame}

\begin{frame}
\frametitle{Conclusion}
\pause
\begin{itemize}
  \item Usage of model can result in expedited time to find new superconductors
  \pause
  \item Can be used in conjunction with regression model made on this data (Hamidieh 2018) to find high-temperature superconductors
  \pause
  \item Further studies can test the accuracy of this model by experimentally verifying superconductivity in the materials predicted
\end{itemize}



\end{frame}

\section{References}
\begin{frame}
\frametitle{Outline}
\tableofcontents[currentsection]
\end{frame}

\begin{frame}
\frametitle{References}
\end{frame}


\section{Acknowledgements}
\begin{frame}
\frametitle{Outline}
\tableofcontents[currentsection]
\end{frame}

\begin{frame}
\frametitle{Acknowledgements}
I would like to thank Leo Lo and Dr. Serena McCalla for their mentorship through the iResearch Institute.

\begin{center}
\includegraphics[scale = 0.58]{iresearch.png}
\end{center}

I would also like to acknowledge my parents for their constant support.
\end{frame}









\end{document}
